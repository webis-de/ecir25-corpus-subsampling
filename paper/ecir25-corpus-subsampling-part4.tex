\section{Evaluation}

\subsection{Theoretical Validation of Corpus Subsampling}

leave-one-group-out on run + qrel files

\begin{table*}[t]
\caption{Overview of TBD. We report rank correlations for scenarios without post judgments ($\rho$ and $\tau$) and with post judgments ($\rho_{J}$ and $\tau_{J}$). {\color{red} TODO?We report statistical significance according to Student's t-test with Bonferroni correction at p=0.05 to the na\"{i}ve lower ($\dagger$) and upper bound ($\ddagger$), respectively condensed lists ($\ast$).}}
\label{table-theoretical-validation}
\renewcommand{\tabcolsep}{3.5pt} 
\centering
\footnotesize

\begin{tabular}{@{}ll@{\hspace{1.5em}}|cc@{\hspace{1.5em}}cc|@{\hspace{1.5em}}cccc@{}}
\toprule
  & \bfseries Approach     &                                              \multicolumn{4}{c@{\hspace{1em}}}{\bfseries System Ranking Corr.}                                              &                                                     \multicolumn{3}{c}{\bfseries Corpus Statistics}                                                      \\
\cmidrule(l@{\tabcolsep}r@{1em}){3-6} \cmidrule(l@{\tabcolsep}r@{1em}){7-9}
                         &                                           &                     $\tau$                      &                      $\rho$                      &                      $\tau_{J}$                      &                     $\rho_{J}$                                           &                      Docs                      &                     Rel.                      &                      Unj.                      \\
\midrule


\multirow{5}{*}{R04} & Judgment Pool\\


& Re-Rank$_{100}$ & 0.00 & 0.00 & 0.00 & 0.00 & 1221 & 211 & 21 \\
& Re-Rank$_{1000}$ &\\

& Pool$_{100}$ &\\
& Pool$_{1000}$ &\\

\midrule
\multirow{5}{*}{CW09} & Judgment Pool\\


& Re-Rank$_{100}$ & 0.00 & 0.00 & 0.00 & 0.00 & 1221 & 211 & 21 \\
& Re-Rank$_{1000}$ &\\

& Pool$_{100}$ &\\
& Pool$_{1000}$ &\\

\midrule
\multirow{5}{*}{CW12} & Judgment Pool\\


& Re-Rank$_{100}$ & 0.00 & 0.00 & 0.00 & 0.00 & 1221 & 211 & 21 \\
& Re-Rank$_{1000}$ &\\

& Pool$_{100}$ &\\
& Pool$_{1000}$ &\\


\midrule
\multirow{5}{*}{DL} & Judgment Pool\\


& Re-Rank$_{100}$ & 0.00 & 0.00 & 0.00 & 0.00 & 1221 & 211 & 21 \\
& Re-Rank$_{1000}$ &\\

& Pool$_{100}$ &\\
& Pool$_{1000}$ &\\

\end{tabular} 
\vspace*{-1ex}
\end{table*}


{\color{red} TODO: use the measure mentioned by Guido that takes into account score ties etc.

The corpus graph against all relevant documents might also serve as suitable baseline for corpus subsampling.

Furthermore, static corpus filtering, e.g., removal of noise documents.
}

\subsection{The Effect of First-Stage Retrievers}

Green IR Experiments here.
Maybe 50\% from the corpus subsample, 50\% random

\subsection{Cascading Re-Ranking Pipelines}

As many retrieval pipelines are cascading, we also test how corpus subsampling impacts the resource consumption of cascading re-ranking pipelines. Given that corpus subsampling directly impacts the resource consumption for the first stage, it substantially impacts also the impact of re-ranking pipelines, as is becomes more simple to show that subsequent stages are more robust for switching the first stage re-ranker. E.g., the approach for efficient xy means z.

Pipeline:
- run multiple first stages
- combine union of first stage outputs, score all of them
- simulate re-ranking via tabulation

{\color{red} GreenIR Experiments. extrapolate what amount of resources would be saved.}
